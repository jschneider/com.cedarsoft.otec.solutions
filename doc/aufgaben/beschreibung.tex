\documentclass[oneside,a4paper]{scrartcl}

\input{/home/johannes/documents/cedarsoft/korrespondenz/vorlagen/tex/common.tex}


\author{Johannes Schneider}

\begin{document}


\centerline{\sc \large Aufgabenbeschreibung Live-Aufgabe}
\vspace{.5pc}
\centerline{\sc Objekttechnologien - Wintersemester 2013/14}
\vspace{2pc}



\section{Grundlegendes}

Im Rahmen dieser Live-Aufgabe soll mit Hilfe des Composite-Patterns ein 
sehr stark vereinfachter Bug-Tracker abgebildet und in Java umgesetzt werden:


\section{Erstellen Sie folgende Klassen/Interfaces}


\subsection{Bug}
Entspricht einem einfachen Bug-Eintrag in unserem Tracker. Der Einfachheit halber enthält
der Bug einen einfache String in dem die Beschreibung (\enquote{description}) abgelegt ist.

Außerdem noch das Feld \enquote{int estimate;} welches den
geschätzen Zeitbedarf (in Stunden) für den Bug repräsentiert.



\subsection{Comment}
Diese Klasse entspricht einem (stark vereinfachten) Kommentar. Der Kommentar enthält ebenfalls
nur einen einfachen String, welcher den Text des Kommentars repräsentiert.


\subsection{BugTracker}
Ein BugTracker enthält das Field \enquote{children} in welches Comments und Bugs eingetragen werden können.


\section{Composite-Pattern}
Wenden Sie nun das Composite Pattern an um die Hierarchie flexibler zu gestalten und die gemeinsame Funktionalität bereit zu stellen.

Führen Sie dazu folgendes ein:

\subsection{BugTrackerEntity (Interface)}
Dieses Interface bietet die gemeinsame Funktionalität aller in der Hierarchie vorhandener Klassen.

Dies sind folgende beiden Methoden:

\begin{description}
	\item[int getEstimation()]
		Liefert den geschätzen Zeitbedarf für das einzelne Entity innerhalb des Bugtrackers - also ohne die Zeiten der (eventuell vorhandenen Kindern).
		Comment und BugTracker selbst haben dabei immer einen \enquote{estimate} von \enquote{0}
	\item[int getTotalEstimation()]		
		Liefert den geschätzten Gesamt-Aufwand - also inklusive der Zeiten aller Kinder und Kindeskinder usw.
\end{description}



\subsection{Hierarchie}
Die Klassen BugTracker, Bug und Comment sollen dabei in die Hierarchie aufgenommen werden.
Dabei soll (mindestens) folgendes möglich sein:

\begin{description}
	\item[BugTracker enthält Bugs]
	\item[Bugs enthalten weitere Bugs] Diese könnten verwandte/abhängige Bugs repräsentieren.
	\item[Bugs enthalten Comments]		
	\item[Comments enthalten weitere Comments]		
\end{description}


\subsubsection{Wichtiger Hinweis}
Sie müssen explizit *nicht* ausschließen, dass die Hierarchie auch fachlich \enquote{unsinnig} aufgebaut wird.
D.h. wenn der Benutzer z.B. einen BugTracker als Kind eines Comments hinzufügt, ist das ein fachliches Problem, welches nicht technisch verhindert werden muss.


\section*{Hinweise}
\subsection*{Unit-Tests}
Erstellen Sie ausreichend Unit-Tests mit den entsprechenden Assertions um die Korrektheit Ihrer Lösung
beweisen zu können.

\subsection*{Composite-Pattern nutzen}
Die Lösung ist nur dann korrekt, wenn sie mit Hilfe des Composite-Patterns umgesetzt wurde. Alternative Lösungen
ohne das Composite-Pattern sind nicht gestattet. 

\subsection*{Encoding}
Bitte achten Sie darauf, dass Ihre Lösung im UTF-8-Encoding abgegeben wird.


\end{document}

