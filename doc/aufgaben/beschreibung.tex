\documentclass[oneside,a4paper]{scrartcl}

\input{/home/johannes/documents/cedarsoft/korrespondenz/vorlagen/tex/common.tex}


\author{Johannes Schneider}

\begin{document}


\centerline{\sc \large Aufgabenbeschreibung Live-Aufgabe}
\vspace{.5pc}
\centerline{\sc Objekttechnologien - Sommersemester 2015}
\vspace{2pc}



\section{Grundlegendes}

Im Rahmen dieser Live-Aufgabe soll mit Hilfe des Command-Patterns ein Joystick für ein Computer-Spiel simuliert werden:


\section{Erstellen Sie folgende Klassen/Interfaces}
\subsection{Klasse Joystick}

Ein Joystick enthält vier Methoden, die je dem Drücken eines Knopfes entsprechen:


\begin{itemize}
\item buttonAPressed
\item buttonBPressed
\item buttonXPressed
\item buttonYPressed
\end{itemize}

Jede dieser Methode gibt beim Aufruf eine kurze Ausgabe auf der Konsole aus. Damit ist erkennbar, dass der entsprechende Button gedrückt wurde.


\section{Command-Pattern}

\subsection{Command-Interface und Implementierungen}

Erstellen Sie ein Command-Interface und folgende drei Implementierungen:

\begin{itemize}
\item JumpCommand
\item MoveCommand
\item ShootCommand
\end{itemize}

Jede dieser drei Implementierungen gibt auf der Konsole eine entsprechende Meldung aus, sobald das
Command ausgeführt wird.


\subsection{Verwendung im Joystick}
In der Joystick-Klasse soll nun jedem Button ein Command zugeordnet werden.
Dieses Kommand wird dann beim Drücken des Knopfs ausgeführt.

Diese Zuordnung erfolgt von außerhalb der Joystick-Klasse.

In unserem Fall wird dasselbe Kommando bei jedem Drücken des jeweiligen Knopfes ausgeführt. D.h.
das selbe Kommandos wird bei mehrmaligem Drücken eines Knopfes auch mehrfach ausgeführt.

\subsection{Speichern der ausgeführten Kommandos}

Die Joystick-Klasse merkt sich die ausgeführten Kommandos. Sie stellt diese Liste per Getter zur Verfügung.

Der Test/Main-Klasse gibt nach dem Drücken der Knöpfe die Liste der ausgeführten Kommandos auf der Konsole
aus.


\subsection{Unit-Test / Main-Methode}
Erstellen Sie einen Unit-Test oder eine Main-Klasse welches das Verhalten des Command-Patterns verdeutlicht.



\end{document}
