\documentclass[oneside,a4paper]{scrartcl}

\input{/home/johannes/documents/cedarsoft/korrespondenz/vorlagen/tex/common.tex}


\author{Johannes Schneider}

\begin{document}


\centerline{\sc \large Aufgabenbeschreibung Live-Aufgabe}
\vspace{.5pc}
\centerline{\sc Objekttechnologien - Sommersemester 2014 / 2}
\vspace{2pc}



\section{Grundlegendes}

Im Rahmen dieser Live-Aufgabe soll mit Hilfe des Observer-Patterns die aus der Fußball-WM bekannte Goal-Line-Überwachungs-Technik abgebildet und in Java umgesetzt werden:


\section{Erstellen Sie Klassen/Interfaces für folgende Funktionalität}

Die Torline jedes Tores bei den WM-Spielen wird mit Hilfe von Kameras überwacht. Auf diesem Weg wird
erkannt, wenn der Ball die Torlinie überquert.

Diese Information wird dann an unterschiedliche Empfänger weiter geleitet:

\begin{itemize}
\item Uhr des Schiedsrichters
\item Logbuch des Systems
\item Regieraum (für die Fernsehübertragung)
\end{itemize}


\subsection{Schiedsrichter-Uhr}
Diese Uhr zeigt das Ereignis einfach dar (Simulation über Konsolenausgabe). Wichtig:
Erst wenn der Schiedsrichter ein Tor quittiert hat (einfacher Methoden-Aufruf) , wird ein erneutes Überqueren der Torlinie angezeigt.

\subsection{Logbuch}
Das Logbuch enthält eine Liste mit den jeweiligen Zeitpunkten, an denen der Ball die Linie überquert hat.

\subsection{Regieraum}
Hier erfolgt jedesmal eine Ausgabe (auf der Konsole). Hier ist auch keine Rückstellung oder ähnliches notwendig.


\end{document}

