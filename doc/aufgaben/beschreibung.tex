\documentclass[oneside,a4paper]{scrartcl}

\input{/home/johannes/documents/cedarsoft/korrespondenz/vorlagen/tex/common.tex}


\author{Johannes Schneider}

\begin{document}


\centerline{\sc \large Aufgabenbeschreibung Live-Aufgabe}
\vspace{.5pc}
\centerline{\sc Objekttechnologien - Sommersemester 2014}
\vspace{2pc}



\section{Grundlegendes}

Im Rahmen dieser Live-Aufgabe soll mit Hilfe des Observer-Patterns ein 
sehr stark vereinfachtes Roulette-Spiel abgebildet und in Java umgesetzt werden:


\section{Erstellen Sie folgende Klasse}


\subsection{Roulette}
Entspricht dem Roulette-Spiel. 

Diese Klasse enthält die Methode \enquote{int nextSpin()} welche eine Spielrunde
simmuliert. Diese Methode liefert eine Zufallszahl von 0 bis 36 (jeweils inklusive)
zurück, welche die Zahl repräsentiert, auf der die Kugel liegen blieb.


\section{Observer-Pattern}
Es gibt mehrere Parteien, die sich für die Ergebnisse der einzelnene Durchgänge interessieren. 
Diese sollen über das Resultat eines jeden Spins jeweils per Observer-Pattern informiert werden.

\subsection{Spieler}
Es gibt einen Spieler, der immer auf die Nummer 7 setzt und sich freut
wenn diese Zahl getroffen wird. 
Dieser Freude verleiht er mit einer Ausgabe auf \enquote{System.out} Ausdruck,
sobald er darüber informiert wurde.

\subsection{Das Haus}
Das Haus möchte sicher stellen, dass die Roulette-Räder gleichverteilte Resultate erzielen und keine
Zahlen gehäuft auftreten.

Aus diesem Grund wird die Häufigkeit des Auftretens der einzelnen Zahlen mit protokolliert.

\subsection{Programmablauf}
Instantiieren Sie (mit Hilfe einer main-Methode oder eines JUnit-Tests) ein Roulette, registrieren
Sie die beiden Listener \enquote{Spieler} und \enquote{Haus} und führen Sie 100 Spiel-Runden durch.

Anschließend geben Sie das Ergebnis der Protokollierung des \enquote{Hauses} aus.
Dies könnte z.B. in einer einfachen Auflistung in folgender Form erfolgen.

\noindent
0: \{Häufigkeit\}

\noindent
1: \{Häufigkeit\}

\noindent
usw.


\section*{Hinweise}
\subsection*{Tipp bezüglich der Datenstruktur}
Am einfachsten scheint die Speicherung der Häufigkeit in Form eines Arrays zu sein bei
der die erzielte Zahl dem Index des Arrays entspricht.
Andere Datenstrukturen sind aber natürlich ebenfalls möglich.

\subsection*{Hinweis: Zufallszahlen}
Am einfachsten lassen sich Zufallszahlen mit Hilfe der statischen Methode \enquote{random} aus der
Klasse \enquote{java.lang.Math} generieren. Diese Methode liefert einen pseudozufälligen double-Wert zwischen 
0.0 (inklusiv) und 1.0 (exklusiv) zurück.

\subsection*{Unit-Tests}
Erstellen Sie ausreichend Unit-Tests mit den entsprechenden Assertions um die Korrektheit Ihrer Lösung
beweisen zu können.

\subsection*{Observer-Pattern nutzen}
Die Lösung ist nur dann korrekt, wenn sie mit Hilfe des Observer-Patterns umgesetzt wurde. Alternative Lösungen
ohne das Observer-Pattern sind nicht gestattet. 

\subsection*{Encoding}
Bitte achten Sie darauf, dass Ihre Lösung im UTF-8-Encoding abgegeben wird.

\subsection*{Abgabe}
Die Abgabe der Lösung erfolgt über das Relax. Bitte laden Sie Ihre Lösung (nur den \enquote{src}-Folder)
in einem ZIP-File hoch.


\end{document}

