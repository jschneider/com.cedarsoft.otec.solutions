\documentclass[oneside,a4paper]{scrartcl}

\input{/home/johannes/documents/cedarsoft/korrespondenz/vorlagen/tex/common.tex}

\usepackage{listings}

\author{Johannes Schneider}

\begin{document}


\centerline{\textsc{\LARGE Aufgabenbeschreibung Live-Aufgabe}}

\vspace{.5pc}
\centerline{\textsc{\large Objekttechnologien - Wintersemester 2017/18}}
\vspace{2pc}



\section{Grundlegendes}

Im Rahmen dieser Live-Aufgabe soll mit Hilfe des Decorator-Patterns ein (sehr einfacher) Video-Downloader mit Cache-Funktion erstellt werden.


\section{Erstellen Sie folgende Klassen/Interfaces}
\subsection{DefaultVideoDownloader}

Diese Klasse erlaubt es Videos herunter zu laden. In unserem Fall werden die Videos
durch einfache Strings repräsentiert.

Dazu erhält sie folgende Methode:

\subsubsection{\emph{String downloadVideo(int videoId)}}

Diese Methode liefert ein \enquote{Video} in Form eines kurzen Strings zurück. Für jede
unterschiedliche Video-Id wird ein entsprechend unterschiedlicher String zurück
gegeben.

So wird z.B. für die \emph{videoId} 7 folgender String zurück gegeben werden: \enquote{Video-Content für ID 7}.


\section{Decorator-Pattern}

\subsection{Interface VideoDownloader}
Erstellen Sie für die erzeugte Klasse und deren Methode ein Interface mit dem Namen \emph{VideoDownloader}


\subsection{VideoCache}
Erstellen Sie eine Klasse \emph{VideoCache} die als Decorator für \emph{VideoDownloader} dient.

Dieser Decorator speichert für die jeweils letzte Anfrage die Video-ID sowie den Video-String.
Wenn in direkter Folge eine Anfrage zur selben Video-ID erfolgt, soll der gespeicherte Video-String zurück geliefert werden.

\newpage
\noindent
Nochmals zur Verdeutlichung:

\subsubsection{Fall 1: Kein oder falsches Video gespeichert}
Dann wird das dekorierte Objekt nach dem Video gefragt. Dieses Video sowie die zugehörige Video-ID wird im Dekorator
als Field gespeichert.

\subsubsection{Fall 2: Selbe Video-ID wird ein zweites mal angefragt}
Erfolgt eine Anfrage mit einer Video-ID, die der aktuell gespeicherten Video-ID entspricht,
wird das gespeicherte Video direkt zurück geliefert. Das dekorierte Objekt wird dabei nicht angefragt.


\subsection{Test / Main-Klasse}
Erzeugen Sie ein Demo-Programm oder einen JUnit-Test, welcher die Funktionsfähigkeit und Korrektheit Ihrer Lösung demonstriert.


\end{document}

