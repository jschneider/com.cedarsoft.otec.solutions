\documentclass[oneside,a4paper]{scrartcl}

\input{/home/johannes/documents/cedarsoft/korrespondenz/vorlagen/tex/common.tex}


\author{Johannes Schneider}

\begin{document}


\centerline{\sc \large Aufgabenbeschreibung Live-Aufgabe}
\vspace{.5pc}
\centerline{\sc Objekttechnologien - Wintersemester 2012/13}
\vspace{2pc}




Im Rahmen dieser Live-Aufgabe soll folgendes abgebildet und in Java umgesetzt werden:




\section{Erstellen Sie folgende Klassen}
und die entsprechenden Unit-Tests.

\subsection{Transaction}
Diese Klasse entspricht einer (stark vereinfachten) Konten-Bewegung: Sie bezieht
sich in dieser Aufgabe immer auf ein einzelnes Konto (ist also kein Buchungssatz o.\,ä.).

Eine Transaktion hat dabei folgende Eigenschaften:

\begin{description}
	\item[Description]
		Eine einfache Beschreibung in Form eines Strings
	\item[Value]		
		Der Wert der Transaktion als Zahlenwert - dieser kann positiv oder negativ sein.		
\end{description}


\subsection{DefaultAccount}
Objekte dieser Klasse entsprechen einem Bankkonto. Sie haben folgende Eigenschaften:

\begin{description}
	\item[Customer]
		Ein String, welcher den Namen des Kunden enthält.
	\item[Transactions]		
		Jedes Konto enthält eine Liste von Transaktionen. Anhand dieser Transaktionen lässt sich
dann der aktuelle Saldo berechnen.
\end{description}


\subsubsection*{Methoden}

Jedem Konto können über die Methode \enquote{addTransaction} neue Transaktionen hinzugefügt werden.
Der aktuelle Saldo eines Kontos kann über die Methode \enquote{calculateSaldo} abgefragt werden.



\section{Decorator-Pattern}
Wenden Sie nun das Decorator-Pattern an um folgende Funktionalität zu einem Bank-Konto hinzufügen zu können:

\subsection{Transaktionen mit negativem Wert nur bei ausreichender Deckung}
Es soll verhindert werden können, dass ein Konto jemals einen negativen Saldo aufweist.
Dies soll dadurch erreicht werden, dass Transaktionen, durch welche das Konto einen negativen Saldo aufwiese,
erst gar nicht zu einem Konto hinzugefügt werden können.

Der Versuch des Hinzufügens soll durch das Werfen einer (von Ihnen zu erstellenden) 
\enquote{TransactionDeclinedException} abgebrochen werden.



\section*{Hinweise}
\subsection*{Unit-Tests}
Erstellen Sie ausreichend Unit-Tests mit den entsprechenden Assertions um die Korrektheit Ihrer Lösung
beweisen zu können.

\subsection*{Decorator-Pattern nutzen}
Die Lösung ist nur dann korrekt, wenn sie mit Hilfe des Decorator-Patterns umgesetzt wurde. Alternative Lösungen
ohne das Decorator-Pattern sind nicht gestattet. 

\subsection*{Encoding}
Bitte achten Sie darauf, dass Ihre Lösung im UTF-8-Encoding abgegeben wird.


\end{document}

