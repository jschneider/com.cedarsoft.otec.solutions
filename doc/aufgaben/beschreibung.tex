\documentclass[oneside,a4paper]{scrartcl}

\input{/home/johannes/documents/cedarsoft/korrespondenz/vorlagen/tex/common.tex}

\usepackage{listings}

\author{Johannes Schneider}

\begin{document}


\centerline{\textsc{\LARGE Aufgabenbeschreibung Live-Aufgabe}}

\vspace{.5pc}
\centerline{\textsc{\large Objekttechnologien - Sommersemeter 20118}}
\vspace{2pc}



\section{Grundlegendes}

Im Rahmen dieser Live-Aufgabe soll mit Hilfe des Decorator-Patterns eine Variante des Spiels \enquote{FizzBuzz} nachprogrammiert werden.


\section{Erstellen Sie folgende Klasse}

\subsection{DefaultCounter}

Diese Klasse besitzt eine Methode:

\subsubsection{\emph{int nextValue()}}

Diese Methode liefert bei  jedem Aufruf
einen um 1 erhöhten Wert zurück.



\section{Decorator-Pattern}

\subsection{Counter}

Erstellen Sie für die erzeugte Klasse und deren Methode ein Interface mit dem Namen \emph{Counter}


\subsection{Mod3Decorator}
Erstellen Sie eine Klasse \emph{Mod3Decorator}, die als Decorator für \emph{Counter} dient.

Dieser Decorator zeigt folgendes Verhalten:

\begin{itemize}
\item Wenn der vom dekorierten Objekt zurück gelieferte Wert durch \emph{3} teilbar ist, wird ein Text auf der Konsole ausgegeben.
\item Andernfalls erfolgt keine Ausgabe.
\end{itemize}

 Der Wert wird immer unverändert zurück gegeben.


\subsection{Mod4Decorator}
Erstellen Sie eine Klasse \emph{Mod4Decorator}, die als Decorator für \emph{Counter} dient.

Dieser Decorator zeigt folgendes Verhalten:

\begin{itemize}
\item Wenn der vom dekorierten Objekt zurück gelieferte Wert durch \emph{4} teilbar ist, wird ein Text auf der Konsole ausgegeben.
\item Andernfalls erfolgt keine Ausgabe.
\end{itemize}

Der Wert wird immer unverändert zurück gegeben.



\subsection{Mod3And4Decorator}
Erstellen Sie eine Klasse \emph{Mod3And4Decorator}, die als Decorator für \emph{Counter} dient.

Dieser Decorator ändert die von \emph{int nextValue()} zurück gelieferten Werte nach
folgendem Schema ab:

\begin{itemize}
\item Wenn der vom dekorierten Objekt zurück gelieferte Wert durch \emph{3} und \emph{4} teilbar ist, wird \emph{-1} zurück gegeben.
\item Andernfalls wird der Wert unverändert zurück gegeben.
\end{itemize}


\subsection{Ausgabe}
Fügen Sie die einzelnen Decorator in sinnvoller Weise so zusammen, rufen Sie die Methode \emph{nextValue()}
24 mal auf und geben Sie den zurück gelieferten Wert auf der Konsole aus.

Dabei soll folgende Ausgabe entstehen:

\begin{lstlisting}[language=java]
1
2
3 ist durch 3 teilbar
3
4 ist durch 4 teilbar
4
5
6 ist durch 3 teilbar
6
7
8 ist durch 4 teilbar
8
9 ist durch 3 teilbar
9
10
11
12 ist durch 3 teilbar
12 ist durch 4 teilbar
-1
13
14
15 ist durch 3 teilbar
15
16 ist durch 4 teilbar
16
17
18 ist durch 3 teilbar
18
19
20 ist durch 4 teilbar
20
21 ist durch 3 teilbar
21
22
23
24 ist durch 3 teilbar
24 ist durch 4 teilbar
-1
\end{lstlisting}





\section{Zur Klarstellung}

\subsection{Modulo}
Um festzustellen, ob eine Zahl durch eine bestimmte Zahl teilbar ist, bietet sich der Modulo-Operator (\emph{\%}) an.


\noindent
Beispiel:

\begin{lstlisting}[language=java]
if (value % 17 ==0 ){
  //Value ist durch 17 teilbar
}
\end{lstlisting}


\subsection{Test / Main-Klasse}
Erzeugen Sie ein Demo-Programm oder einen JUnit-Test, welcher die Funktionsfähigkeit und Korrektheit Ihrer Lösung demonstriert.


\end{document}

